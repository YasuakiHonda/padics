%% 
%% Copyright 2019 Elsevier Ltd
%% 
%% This file is part of the 'CAS Bundle'.
%% --------------------------------------
%% 
%% It may be distributed under the conditions of the LaTeX Project Public
%% License, either version 1.2 of this license or (at your option) any
%% later version.  The latest version of this license is in
%%    http://www.latex-project.org/lppl.txt
%% and version 1.2 or later is part of all distributions of LaTeX
%% version 1999/12/01 or later.
%% 
%% The list of all files belonging to the 'CAS Bundle' is
%% given in the file `manifest.txt'.
%% 
%% Template article for cas-sc documentclass for 
%% single column output.


\documentclass[fleqn]{cas-sc}

\usepackage[numbers]{natbib}
%%%Author macros
\usepackage{rotating}

\makeatletter
\renewcommand*\env@matrix[1][\arraystretch]{%
  \edef\arraystretch{#1}%
  \hskip -\arraycolsep
  \let\@ifnextchar\new@ifnextchar
  \array{*\c@MaxMatrixCols c}}
\makeatother



\begin{document}
\let\WriteBookmarks\relax
\def\floatpagepagefraction{1}
\def\textpagefraction{.001}
\shorttitle{Computations with $p-$adic numbers in Maxima}
\shortauthors{JA Vallejo}

\title[mode=title]{Computations with $p-$adic numbers in Maxima}                      


\author[1]{JA Vallejo}[orcid=0000-0002-9508-1549]
\cormark[1]
\fnmark[1]
\ead{jvallejo@fc.uaslp.mx}
\ead[URL]{http://galia.fc.uaslp.mx/~jvallejo}
\address[1]{Facultad de Ciencias, Universidad Aut\'onoma de San Luis Potos\'i,\\
Av Chapultepec 1570 Col. Lomas del Pedregal\\
CP 78295 San Luis Potos\'i (SLP) M\'exico}


\cortext[cor1]{Corresponding author}

\fntext[fn1]{Supported by a research grant from the Consejo Nacional de Ciencia y 
Tecnolog\'ia (M\'exico) code A1-S-19428.}


\begin{abstract}
This is just a first attempt to create a Maxima package for working with $p-$adic numbers. It is extremely ugly and slow, lacking anything remotely resembling elegance or optimization, but at least it gives correct answers (for all those examples that I have been able to find in the literature). The current version is suitable for basic courses on the subject of $p-$adic analysis, and maybe for constructing examples of some simple theoretical ideas. As general references, the reader should refer to the excellent books by Bachman \cite{1} and Gouv\^ea
\cite{12}.
\end{abstract}


\begin{keywords}
$p-$adic numbers\\
Hensel codes\\
Finite-segment arithmetic\\
Maxima CAS\\
Symbolic computations
\end{keywords}


\maketitle


\section{Loading the package}

The most simple way to do it consists in putting a copy of \texttt{padics.mac} in your working 
directory. If a Maxima session is launched from that directory, you can load the package with
\begin{verbatim}
(%i1)	load("padics.mac");
(%o1)	"padics.mac"
\end{verbatim}

The other option is to do a global installation, putting a copy of \texttt{padics.mac} in 
the system directory (you will need root permissions for that) \texttt{/usr/share/maxima/5.42.1/share/contrib} or its Windows equivalent. Then, load the package with the same command above.


\section{The $p-$adic norm}\label{sec2}

We can compute the p-adic order of a rational number with \texttt{padic\_order}.
The syntax of all the commands in the package is quite intuitive; for example,
to compute the order of a rational with respect to the prime $p$, we use \texttt{padic\_order(rational,prime)}:

\begin{verbatim}
(%i2)	padic_order(144,3);
(%o2)	2
(%i3)	padic_order(17,3);
(%o3)	0
\end{verbatim}

We follow the convention that the order of $0$ is always (real) infinity:

\begin{verbatim}
(%i4)	makelist(padic_order(0,i),i,[2,3,5,7,11,13]);
(%o4)	[inf,inf,inf,inf,inf,inf]
(%i5)	padic_order(3/10,5);
(%o5)	-1
(%i6)	padic_order(36015/88,7);
(%o6)	4
\end{verbatim}

Notice that the $p-$adic order is an even function:
\begin{verbatim}
(%i7)	padic_order(-3/10,5);
(%o7)	-1
\end{verbatim}

The reduction to the canonical form of a rational number
$$
r=\frac{a}{b}=p^{\mathrm{porder}(r)} \frac{a'}{b'}
$$
can be achieved with \texttt{padic\_canonical}. The result has the form of a list 
\texttt{[p\^{}porder(r),a'/b']}:
\begin{verbatim}
(%i8)	padic_canonical(0.234,2);
(%o8)	[1/4,117/125]
(%i9)	padic_canonical(0,3);
(%o9)	[1,0]
\end{verbatim}

The $p-$adic norm of a rational number is computed by \texttt{padic\_norm}:
\begin{verbatim}
(%i10)	makelist(padic_norm(0,j),j,[2,3,5,7,11,13,17]);
(%o10)	[0,0,0,0,0,0,0]
(%i11)	padic_norm(17,17);
(%o11)	1/17
(%i12)	padic_norm(144,3);
(%o12)	1/9
(%i13)	padic_norm(12,5);
(%o13)	1
(%i14)	makelist(padic_norm(162/13,k),k,[3,13]);
(%o14)	[1/81,13]
\end{verbatim}

The next example comes from \url{http://mathworld.wolfram.com/p-adicNorm.html}:
\begin{verbatim}
(%i15)	makelist(padic_norm(140/297,k),k,[2,3,5,7,11]);
(%o15)	[1/4,27,1/5,1/7,11]
\end{verbatim}

Another example, this one from \url{www.asiapacific-mathnews.com/03/0304/0001_0006.pdf}:
\begin{verbatim}
(%i16)	makelist(padic_norm(63/550,k),k,[2,3,5,7,11,13]);
(%o16)	[2,1/9,25,1/7,11,1]
(%i17)	padic_norm(0.234,2);
(%o17)	4
\end{verbatim}

Let us check the triangle \emph{equality}:
\begin{verbatim}
(%i18)	padic_norm(3/10,5);
(%o18)	5
(%i19)	padic_norm(40,5);
(%o19)	1/5
(%i20)	padic_norm(3/10-40,5);
(%o20)	5
\end{verbatim}

The next examples come from \url{https://www.sangakoo.com/en/unit/p-adic-distance}:
\begin{verbatim}
(%i21)	padic_norm(10/12,2);
(%o21)	2
(%i22)	padic_norm(10/12,5);
(%o22)	1/5
(%i23)	padic_norm(10/12,7);
(%o23)	1
\end{verbatim}

Of course, once we have a norm available, we can define the
associated $p-$adic distance, here denoted \texttt{padic\_distance}.
The first example computes the distance between $2$ and $28814$ in
$\mathbb{Q}_7$ (so you can easily guess the syntax):
	
\begin{verbatim}
(%i24)	padic_distance(2,28814,7);
(%o24)	1/2401
(%i25)	padic_distance(2,3,7);
(%o25)	1
(%i26)	padic_distance(2166^2,2,7);
(%o26)	1/2401
(%i27)	padic_distance(3,3+29^4,29);
(%o27)	1/707281
\end{verbatim}
The next example comes from \url{https://www.sangakoo.com/en/unit/p-adic-distance}:

\begin{verbatim}
(%i28)	padic_distance(82,1,3);
(%o28)	1/81
\end{verbatim}

\section{$p-$adic Expansions (Hensel codes)}\label{sec3}

\noindent As in the case of real numbers, the only explicit representation of 
$p-$adic numbers we can get are those based on rational numbers (periodic $p-$adic
expansions). Here we consider Hensel (pseudo)codes, which basically are truncations
of the $p-$adic expansions to a given order $r$.
The syntax is \texttt{hensel(rational,p,r)}, the output has the form \texttt{[[exponent],mantissa]}, and the algorithm used
here is based on the one proposed by G. Bachman in \cite{1}.

\begin{verbatim}
(%i29)	hensel(5/7,7,7);
(%o29)	[[-1],5,0,0,0,0,0,0]
(%i30)	hensel(-84,7,9);
(%o30)	[[1],2,5,6,6,6,6,6,6,6]
(%i31)	hensel(8/3,5,9);
(%o31)	[[0],1,2,3,1,3,1,3,1,3]
(%i32)	hensel(3/4,5,4);
(%o32)	[[0],2,1,1,1]
(%i33)	hensel(2/15,5,7);
(%o33)	[[-1],4,1,3,1,3,1,3]
(%i34)	hensel(7/6,5,4);
(%o34)	[[0],2,4,0,4]
(%i35)	hensel(2/7,5,4);
(%o35)	[[0],1,2,1,4]
(%i36)	hensel(1/12,5,7);
(%o36)	[[0],3,4,2,4,2,4,2]
(%i37)	hensel(5/8,5,4);
(%o37)	[[1],2,4,1,4]
(%i38)	hensel(1/2,5,4);
(%o38)	[[0],3,2,2,2]
(%i39)	hensel(1/3,5,4);
(%o39)	[[0],2,3,1,3]
(%i40)	hensel(1/4,5,4);
(%o40)	[[0],4,3,3,3]
(%i41)	hensel(1/4,5,7);
(%o41)	[[0],4,3,3,3,3,3,3]
(%i42)	hensel(1/25,5,4);
(%o42)	[[-2],1,0,0,0]
(%i43)	hensel(-7/8,3,5);
(%o43)	[[0],1,2,1,2,1]
\end{verbatim}

The command \texttt{nicehensel} displays the result in the form commonly
found in textbooks and expository works (this form has the drawback
that, when $p>7$, is is impossible to distinguish between the number $11$
and two consecutive $1$'s, so it will not be used in what follows), that is, 
something like
$$
r = a_{-e}...a_{-1}{.}a_0a_1a_2...
$$
where $e$ is the order of $r$.

\begin{verbatim}
(%i44)	nicehensel(8/3,5,9);
(%o44)	.123131313
(%i45)	nicehensel(8/75,5,9);
(%o45)	12.3131313
(%i46)	nicehensel(3/4,5,4);
(%o46)	.2111
(%i47)	nicehensel(2/15,5,7);
(%o47)	4.131313
(%i48)	nicehensel(1/3,5,7);
(%o48)	.2313131
(%i49)	nicehensel(-1/3,5,7);
(%o49)	.3131313
(%i50)	nicehensel(5/1,5,4);
(%o50)	.0100
(%i51)	nicehensel(25,5,4);
(%o51)	.0010
(%i52)	nicehensel(2/7,5,4);
(%o52)	.1214
\end{verbatim}

Let us compare this with the table presented in \cite{2}:

\begin{verbatim}
(%i53)	h[i,j]:=nicehensel(i/j,5,4)$
(%i54)	genmatrix(h,17,17);
\end{verbatim}
\setcounter{MaxMatrixCols}{20}
\begin{sideways}
\begin{minipage}{\textheight}
\[
\begin{pmatrix}
.1000 & .3222 & .2313 & .4333 & 1.000 & .1404 & .3302 & .2414 & .4201 & 3.222 & .1332 & .3424 & .2034 & .4101 & 2.313 & .1234 & .3043\\
.2000 & .1000 & .4131 & .3222 & 2.000 & .2313 & .1214 & .4333 & .3012 & 1.000 & .2120 & .1404 & .4014 & .3302 & 4.131 & .2414 & .1132\\
.3000 & .4222 & .1000 & .2111 & 3.000 & .3222 & .4021 & .1303 & .2313 & 4.222 & .3403 & .4333 & .1143 & .2013 & 1.000 & .3104 & .4121\\
.4000 & .2000 & .3313 & .1000 & 4.000 & .4131 & .2423 & .3222 & .1124 & 2.000 & .4240 & .2313 & .3123 & .1214 & 3.313 & .4333 & .2210\\
.0100 & .0322 & .0231 & .0433 & .1000 & .0140 & .0330 & .0241 & .0420 & .3222 & .0133 & .0342 & .0203 & .0410 & .2313 & .0123 & .0304\\
.1100 & .3000 & .2000 & .4222 & 1.100 & .1000 & .3142 & .2111 & .4131 & 3.000 & .1411 & .3222 & .2232 & .4021 & 2.000 & .1303 & .3342\\
.2100 & .1322 & .4313 & .3111 & 2.100 & .2404 & .1000 & .4030 & .3432 & 1.322 & .2204 & .1202 & .4212 & .3222 & 4.313 & .2042 & .1431\\
.3100 & .4000 & .1231 & .2000 & 3.100 & .3313 & .4302 & .1000 & .2243 & 4.000 & .3041 & .4131 & .1341 & .2423 & 1.231 & .3222 & .4420\\
.4100 & .2322 & .3000 & .1433 & 4.100 & .4222 & .2214 & .3414 & .1000 & 2.322 & .4324 & .2111 & .3321 & .1134 & 3.000 & .4402 & .2024\\
.0200 & .0100 & .0413 & .0322 & .2000 & .0231 & .0121 & .0433 & .0301 & .1000 & .0212 & .0140 & .0401 & .0330 & .4131 & .0241 & .0113\\
.1200 & .3322 & .2231 & .4111 & 1.200 & .1140 & .3423 & .2303 & .4012 & 3.322 & .1000 & .3020 & .2430 & .4431 & 2.231 & .1421 & .3102\\
.2200 & .1100 & .4000 & .3000 & 2.200 & .2000 & .1330 & .4222 & .3313 & 1.100 & .2332 & .1000 & .4410 & .3142 & 4.000 & .2111 & .1240\\
.3200 & .4322 & .1413 & .2433 & 3.200 & .3404 & .4142 & .1241 & .2124 & 4.322 & .3120 & .4424 & .1000 & .2343 & 1.413 & .3340 & .4234\\
.4200 & .2100 & .3231 & .1322 & 4.200 & .4313 & .2000 & .3111 & .1420 & 2.100 & .4403 & .2404 & .3034 & .1000 & 3.231 & .4030 & .2323\\
.0300 & .0422 & .0100 & .0211 & .3000 & .0322 & .0402 & .0130 & .0231 & .4222 & .0340 & .0433 & .0114 & .0201 & .1000 & .0310 & .0412\\
.1300 & .3100 & .2413 & .4000 & 1.300 & .1231 & .3214 & .2000 & .4432 & 3.100 & .1133 & .3313 & .2143 & .4302 & 2.413 & .1000 & .3401\\
.2300 & .1422 & .4231 & .3433 & 2.300 & .2140 & .1121 & .4414 & .3243 & 1.422 & .2411 & .1342 & .4123 & .3013 & 4.231 & .2234 & .1000
\end{pmatrix}
\]
\end{minipage}
\end{sideways}


\section{Arithmetic with $p-$adics}\label{sec4}

\noindent The basic arithmetic functions are implemented as:
\begin{itemize}
	\item  \texttt{padic\_sum} (for sum, addition)
	\item  \texttt{padic\_substract} (for difference, substraction)
	\item  \texttt{padic\_multiply} (for product, multiplication)
	\item  \texttt{padic\_divide} (for division, quotient)
\end{itemize}

The syntax is quite evident: each function takes the arguments
\texttt{(operand1,operand2,p)}.
Some test numbers:
\begin{verbatim}
(%i55)	l1:hensel(3/10,5,4);
(l1)	[[-1],4,2,2,2]
(%i56)	l2:hensel(1/2,5,4);
(l2)	[[0],3,2,2,2]
(%i57)	padic_sum(l1,l2,5);
(%o57)	[[-1],4,0,0,0]
\end{verbatim}

Notice that the Hensel code for the sum $3/10+1/2$ corresponds to $4/5$:
\begin{verbatim}
(%i58)	hensel(4/5,5,4);
(%o58)	[[-1],4,0,0,0]
\end{verbatim}

Also, notice that a consequence of using a finite segment representation is that
adding up a really small number with a really big one just gives the bigger:
\begin{verbatim}
(%i59)	padic_sum([[2],2,5,1,5],[[-3],3,3,3,2],7);
(%o59)	[[-3],3,3,3,2]
\end{verbatim}

Another test (this is an example in \cite{2}) with $p=5$ and $r=9$
\begin{verbatim}
(%i60)	h1:hensel(2/3,5,9);
(h1)	[[0],4,1,3,1,3,1,3,1,3]
(%i61)	h2:hensel(5/6,5,9);
(h2)	[[1],1,4,0,4,0,4,0,4,0]
(%i62)	padic_sum(h1,h2,5);
(%o62)	[[0],4,2,2,2,2,2,2,2,2]
\end{verbatim}

Let us check the following example in \cite{2}:
\begin{verbatim}
(%i63)	padic_substract(h1,h2,5);
(%o63)	[[0],4,0,4,0,4,0,4,0,4]
\end{verbatim}

And those of \cite{3}:
\begin{verbatim}
(%i64)	padic_substract(hensel(3/4,5,4),hensel(3/2,5,4),5);
(%o64)	[[0],3,3,3,3]
\end{verbatim}

An example on multiplication, also from \cite{3}:
\begin{verbatim}
(%i65)	t1:hensel(4/15,5,4);
(t1)	[[-1],3,3,1,3]
(%i66)	t2:hensel(5/2,5,4);
(t2)	[[1],3,2,2,2]
(%i67)	padic_multiply(t1,t2,5);
(%o67)	[[0],4,1,3,1]
\end{verbatim}

Another example from \cite{2}:
\begin{verbatim}
(%i68)	padic_multiply(h1,h2,5);
(%o68)	[[1],4,2,0,1,2,4,3,2,0]
\end{verbatim}

An example from \cite{4}:
\begin{verbatim}
(%i69)	al:hensel(1/4,5,4);
(al)	[[0],4,3,3,3]
(%i70)	be:hensel(1/3,5,4);
(be)	[[0],2,3,1,3]
(%i71)	padic_multiply(al,be,5);
(%o71)	[[0],3,4,2,4]
\end{verbatim}

The function \texttt{normalize\_hensel} normalizes the Hensel code so that the
first digit after the dot is not zero:
\begin{verbatim}
(%i72)	normalize_hensel([[-1],0,0,1,2,3]);
(%o72)	[[1],1,2,3]
\end{verbatim}
It is internally used by the function for computing divisions. Our first
example is a trivial one:
\begin{verbatim}
(%i73)	padic_divide([[0],4,0,0,0,0,0,0],[[0],2,0,0,0,0,0,0],7);
(%o73)	[[0],2,0,0,0,0,0,0]
\end{verbatim}

The next example is from \cite{2}:
\begin{verbatim}
(%i74)	dividend:[[0],4,1,3,1,3,1,3];
(dividend)	[[0],4,1,3,1,3,1,3]
(%i75)	divisor:[[0],3,4,2,4,2,4,2];
(divisor)	[[0],3,4,2,4,2,4,2]
(%i76)	padic_dividei(dividend,divisor,5);
(%o76)	[[0],3,1,0,0,0,0,0]
(%i77)	d1:[[0],2,1,1,1];
(d1)	[[0],2,1,1,1]
(%i78)	d2:[[-1],1,1,0,0];
(d2)	[[-1],1,1,0,0]
(%i79)	padic_divide(d1,d2,5);
(%o79)	[[1],2,4,1,4]
(%i80)	padic_divide([[0],4,3,3,3],[[0],0,1,4,0],5);
(%o80)	[[-2],0,4,2,2]
\end{verbatim}

This is the example presented in \cite{3}: $1/4/(1/2+1/3)+1/25$
\begin{verbatim}
(%i81)	padic_sum(padic_divide([[0],4,3,3,3],
                          padic_sum([[0],3,2,2,2],[[0],2,3,1,3],5)
                          ,5),
                 [[-2],1,0,0,0],
                 5);
(%o81)	[[-2],1,4,2,2]
\end{verbatim}

A quick check that the answer is correct:
\begin{verbatim}
(%i82)	1/4/(1/2+1/3)+1/25;
(%o82)	17/50
(%i83)	hensel(17/50,5,4);
(%o83)	[[-2],1,4,2,2]
\end{verbatim}

Another example, from \cite{4}:
\begin{verbatim}
(%i84)	alf:hensel(8/9,5,4);
(alf)	[[0],2,2,4,3]
(%i85)	bet:hensel(1/2,5,4);
(bet)	[[0],3,2,2,2]
(%i86)	padic_divide(alf,bet,5);
(%o86)	[[0],4,4,3,2]
\end{verbatim}

We reproduce here one more example, from \cite{6}:
\begin{verbatim}
(%i87)	hensel(1/333333,5,27);
(%o87)	[[0],2,4,4,4,4,4,2,1,2,3,4,4,1,2,3,1,0,1,2,3,2,3,1,3,1,0,2]
(%i88)	time(%);
(%o88)	[0.001]
\end{verbatim}

Reference \cite{6} states that $3$ seconds were required for this computation,
using a C++ library, in $2005$.

\section{From Hensel codes to rationals}\label{sec5}

\noindent Passing from Hensel codes to equivalent rational numbers in $\mathbb{Q}_p$
requires the use of the appropriate Farey fractions. A function for
generating the Farey fractions $\mathbb{I}_n$ is \texttt{farey(n)}:
\begin{verbatim}
(%i89)	farey(17);
(%o89)	[0,1/17,1/16,1/15,1/14,1/13,1/12,1/11,1/10,1/9,2/17,1/8,2/15,1/7,
         2/13,1/6,3/17,2/11,3/16,1/5,3/14,2/9,3/13,4/17,1/4,4/15,3/11,2/7,
         5/17,3/10,4/13,5/16,1/3,6/17,5/14,4/11,3/8,5/13,2/5,7/17,5/12,3/7,
         7/16,4/9,5/11,6/13,7/15,8/17,1/2,9/17,8/15,7/13,6/11,5/9,9/16,4/7,
         7/12,10/17,3/5,8/13,5/8,7/11,9/14,11/17,2/3,11/16,9/13,7/10,12/17,
         5/7,8/11,11/15,3/4,13/17,10/13,7/9,11/14,4/5,13/16,9/11,14/17,5/6,
         11/13,6/7,13/15,7/8,15/17,8/9,9/10,10/11,11/12,12/13,13/14,14/15,
         15/16,16/17,1]
(%i90)	time(%);
(%o90)	[0.0]
\end{verbatim}

The package implements the algorithm by Gregory and Krishnamurthy
described in the book \cite{7}. We have added a heuristic routine
to detect a few particular cases in order to give cleaner results.
For instance, cases such as 
\texttt{[[m],a0,0,0,0,...]},\texttt{[[m],a0,p-1,p-1,p-1,...]} or 
\texttt{[[m],a0,a1,...,ak,0,0,...,0s]}
with $s>k$.

The syntax is \texttt{hensel\_to\_farey(list,p)}, where \texttt{list} is
a Hensel code, and $p$ is the prime we are considering.
Some trivial examples:
\begin{verbatim}
(%i91)	hensel_to_farey([[2],3,0,0,0,0,0,0,0],7);
(%o91)	147
(%i92)	hensel_to_farey([[0],4,4,4,4,4,4,4],5);
(%o92)	-1
(%i93)	hensel_to_farey([[0],3,3,3,3,3,3,3],5);
(%o93)	-3/4
(%i94)	hensel_to_farey([[0],0,0,0,0,0,0],5);
(%o94)	0
\end{verbatim}

From \cite{6} (pg 12):
\begin{verbatim}
(%i95)	hensel_to_farey([[0],2,3,1,5],7);
(%o95)	9/43
\end{verbatim}

From \cite{7} (pp. 100 and ff):
\begin{verbatim}
(%i96)	hensel_to_farey([[0],0,2,4,1],5);
(%o96)	5/8
(%i97)	hensel_to_farey([[1],2,4,1,4],5);
(%o97)	5/8
(%i98)	hensel_to_farey([[-1],3,2,2,2],5);
(%o98)	1/10
(%i99)	hensel_to_farey([[0],3,2,2,2],5);
(%o99)	1/2
(%i100)	hensel_to_farey([[0],2,3,1,3],5);
(%o100)	1/3
(%i101)	hensel_to_farey([[0],0,6,0,6],7);
(%o101)	-7/8
\end{verbatim}

A quick check:
\begin{verbatim}
(%i102)	hensel(1/3,5,4);
(%o102)	[[0],2,3,1,3]
(%i103)	hensel(-7/8,7,4);
(%o103)	[[1],6,0,6,0]
\end{verbatim}

Example from \cite{11}:
\begin{verbatim}
(%i104)	hensel_to_farey([[0],2,2,1,0],5);
(%o104)	4/17
(%i105)	hensel_to_farey([[0],1,2,1,4],5);
(%o105)	2/7
\end{verbatim}

Examples from the paper \cite{8}:
\begin{verbatim}
(%i106)	hensel_to_farey([[0],2,2,1,0],5);
(%o106)	4/17
(%i107)	hensel_to_farey([[0],2,2,3,4],5);
(%o107)	17/16
(%i108)	hensel_to_farey([[0],4,2,3,4],5);
(%o108)	13/17
(%i109)	hensel_to_farey([[0],1,1,0,0,1,0],3);
(%o109)	-13/17
(%i110)	hensel_to_farey([[0],1,2,1,4],5);
(%o110)	2/7
(%i111)	hensel_to_farey([[0],3,4,2,3],5);
(%o111)	11/7
\end{verbatim}

\section{Hensel codes of square roots}\label{sec6}

\noindent The computation of square roots in $\mathbb{Q}_p$ is a very interesting
topic that depends on a lot of number-theoretical notions, among which that of a
quadratic residue is the most important (see \cite{9,12}).
An integer $q$ is called a quadratic residue mod $p$ if there exists an
integer $x$ such that  $x^2 =q \mod p$.

\noindent\textbf{Remark}: If $p=2$, every integer is a quadratic residue. 
If $p$ is a prime different from $2$, there are $(p-1)/2$ residues and 
$(p-1)/2$ non-residues in $\mathbb{F}_p-\{0\}$ (that is, the multiplicative
group of $\mathbb{F}_p$ or $\mathbb{F}^*_p$).

As a consequence of the reciprocity law in Number Theory, we get the following criterion:
\begin{enumerate}[(a)]\renewcommand{\theenumi}{\alph{enumi}}
	\item If $p=1 \mod 4$, then $-1$ is a quadratic residue mod $p$.
	\item If $p=3 \mod 4$, then $-1$ is a non residue mod $p$.
	Notice that every prime is equivalent to $1$ or $3$ mod $4$.
\end{enumerate}

Euler's criterion for quadratic reciprocity states that (given $a \in \mathbb{Z}$
and $p$ an odd prime) the Legendre symbol evaluates to $(a|p)=a^{(p-1)/2}$, 
and this equals $1\mod p$  if $a$ is a quadratic residue and $-1\mod p$ if it is not.

First, we introduce a function \texttt{sqrtmod} to determine whether a given integer
is a quadratic residue modulo $p$ or not. For example, $a=2$ is not a quadratic residue
modulo $p=5$:
\begin{verbatim}
(%i112)	sqrtmod(2,5);
(%o112)	"Not a quadratic residue"
\end{verbatim}
But it is modulo $p=7$:
\begin{verbatim}
(%i113)	sqrtmod(2,7);
(%o113)	[3,4]
\end{verbatim}

If $a$ is a quadratic residue modulo $p$ with root $x$, then another root is
given by the $y$ such that $y=-x \mod p$. We compute the $p-$adic roots numerically,
with the aid of Newton's method, using the command \texttt{padic\_sqrt}, whose syntax is
\texttt{padic\_sqrt(number,p)}. The command admits an optional argument fixing the number
of iterations to be done, as in \texttt{padic\_sqrt(number,p,iterations)}.
\begin{verbatim}
(%i114)	padic_sqrt(2,7);
(%o114)	[215912063945802350977/152672884556058511392,
         2267891697076964737/1603641597827614272]
\end{verbatim}

We can get the corresponding Hensel of the roots codes through
\begin{verbatim}
(%i115)	map(lambda([u],hensel(u,7,9)),%);
(%o115)	[[[0],3,1,2,6,1,2,1,2,4],[[0],4,5,4,0,5,4,5,4,2]]
\end{verbatim}

Another example: the square root of $7$ in $\mathbb{Q}_3$ (from \cite{9}):
\begin{verbatim}
(%i116)	padic_sqrt(7,3,3);
(%o116)	[977/368,108497/41008]
\end{verbatim}

The first fraction contains $3\times 2=6$ exact digits to determine the square
root of $7$ in $\mathbb{Q}_3$. To get its corresponding Farey sequence it does not
make sense to take more than $6$ digits. Hence we do the following:

\begin{verbatim}
(%i117)	map(lambda([u],hensel(u,3,6)),%);
(%o117)	[[[0],1,1,1,0,2,0],[[0],2,1,1,2,0,2]]
(%i118)	hensel_to_farey(%[1],3);
(%o118)	1/25
\end{verbatim}

The result is not exactly the rational we started with, but it is very close
in $\mathbb{Q}_3$:
\begin{verbatim}
(%i119)	padic_distance(977/368,1/25,3);
(%o119)	1/2187
\end{verbatim}

More examples:
\begin{verbatim}
(%i120)	padic_sqrt(6,5)[1];
(%o120)	80746825394092993/32964753427463648
(%i121)	hensel(%,5,4);
(%o121)	[[0],1,3,0,4]
\end{verbatim}

The results can be compared against those given in
the on-line calculator \url{http://www.numbertheory.org/php/p-adic.html}:
\begin{verbatim}
(%i122)	padic_sqrt(25,7);
(%o122)	[552213837122886833247075521/110442767424206762611644736,5]
(%i123)	map(lambda([u],hensel(u,7,8)),%);
(%o123)	[[[0],2,6,6,6,6,6,6,6],[[0],5,0,0,0,0,0,0,0]]
\end{verbatim}

Example from \cite{10}:
\begin{verbatim}
(%i124)	padic_sqrt(-2,3);
(%o124)	[-28545857/22783264,28545857/22783264]
(%i125)	hensel(%[1],3,8);
(%o125)	[[0],1,1,2,0,0,2,0,1]
\end{verbatim}

Again an example from \cite{9}, to see the influence of the choice of initial condition
in Newton's iteration (notice how we fix the number of iterations as $3$ here):
\begin{verbatim}
(%i126)	padic_sqrt(1,3,3);
(%o126)	[1,3281/3280]
(%i127)	map(lambda([u],hensel(u,3,8)),%);
(%o127)	[[[0],1,0,0,0,0,0,0,0],[[0],2,2,2,2,2,2,2,2]]
(%i128)	map(lambda([u],hensel_to_farey(u,3)),%);
(%o128)	[1,-1]
\end{verbatim}

\section{Solving $p-$adic systems of linear equations}\label{sec7}

\noindent This is a topic of fundamental importance in applications. The algorithm
	implemented here is Gaussian reduction, and in order to solve the problem $Ax=b$
	we have the commands \texttt{padic\_gauss} and
	\texttt{padic\_backsub}. The first one triangularizes the system, and its syntax
	is \texttt{padic\_gauss(B,p)}, where $B=A|b$ is the augmented matrix of the system
	(that is, the coefficient matrix $A$ augmented with the column $b$ of non
	homogeneous terms). The resulting triangular matrix is processed
	by \texttt{padic\_backsub} to obtain the Hensel codes of the solution.

The following examples are from \cite{4}:
\begin{verbatim}
(%i129)	D:matrix([3,1,3,16],[1,3,1,8],[1,1,3,12])$
(%i130)	padic_gauss(D,11);
(%o130)	matrix(
		[[[0],3,0,0,0],	[[0],1,0,0,0],	[[0],3,0,0,0],	[[0],5,1,0,0]],
		[[[0],0,0,0,0],	[[0],10,3,7,3],	[[0],0,0,0,0],	[[0],10,3,7,3]],
		[[[0],0,0,0,0],	[[0],0,0,0,0],	[[0],2,0,0,0],	[[0],6,0,0,0]]
	)
(%i131)	padic_backsub(%,11);
(%o131)	[[[0],2,0,0,0],[[0],1,0,0,0],[[0],3,0,0,0]]
\end{verbatim}

By converting to Farey fractions, we get the rational form of the solutions:

\begin{verbatim}
(%i132)	map(lambda([x],hensel_to_farey(x,11)),%);
(%o132)	[2,1,3]
\end{verbatim}

Another example:
\begin{verbatim}
(%i133)	C:matrix([2,2,-1,5],[-3,0,2,-5],[4,-5,-1,0]);
(C)	matrix(
		[2,	2,	-1,	5],
		[-3,	0,	2,	-5],
		[4,	-5,	-1,	0]
	)
(%i134)	padic_gauss(C,5);
(%o134)	matrix(
		[[[0],2,0,0,0,0,0],	[[0],2,0,0,0,0,0],	[[0],4,4,4,4,4,4],	[[1],1,0,0,0,0,0]],
		[[[0],0,0,0,0,0,0],	[[0],3,0,0,0,0,0],	[[0],3,2,2,2,2,2],	[[1],3,2,2,2,2,2]],
		[[[0],0,0,0,0,0,0],	[[0],0,0,0,0,0,0],	[[0],0,3,2,2,2,2],	[[0],0,2,2,2,2,2]]
	)
(%i135)	padic_backsub(%,5);
(%o135)	[[[-2],0,0,1,0,0,0],[[-2],0,0,1,0,0,0],[[-2],0,0,4,4,4,4]]
\end{verbatim}

We convert the solution to rational (Farey) expressions:
\begin{verbatim}
(%i136)	map(lambda([x],hensel_to_farey(x,5)),%);
(%o136)	[1,1,-1]
\end{verbatim}

The examples above were trivial, the only intention was to show
	how the usual (rational) results are recovered within $p-$adic algebra.
	Now we consider more advanced examples, with some matrices that
	are highly unstable in rational arithmetic. As the output are really
	big expressions, we suppress them adding a dollar symbol at the end of the commands:
	
\begin{verbatim}
(%i137)	E:matrix(
	[10,9,8,7,6,5,4,3,2,1,1],
	[9,9,8,7,6,5,4,3,2,1,2],
	[8,8,8,7,6,5,4,3,2,1,-5],
	[7,7,7,7,6,5,4,3,2,1,9],
	[6,6,6,6,6,5,4,3,2,1,15],
	[5,5,5,5,5,5,4,3,2,1,1],
	[4,4,4,4,4,4,4,3,2,1,6],
	[3,3,3,3,3,3,3,3,2,1,14],
	[2,2,2,2,2,2,2,2,2,1,3],
	[1,1,1,1,1,1,1,1,1,1,1]
	)$
(%i138)	padic_gauss(E,8209)$
(%i139)	time(%);
(%o139)	[1.038]
(%i140)	padic_backsub(%th(2),8209)$
(%i141)	map(lambda([x],hensel_to_farey(x,8209)),%);
(%o141)	[-1,8,-21,8,20,-19,-3,19,-9,-1]
\end{verbatim}
Of course, the solution we have obtained is exactly the same that results
when using a built-in command such as \texttt{solve} or \texttt{linsolve}.
This is so because the coefficients of $E$ are rational. However, even 
restricting ourselves to computations with rationals, there are cases where
the use of $p-$adic arithmetic can be convenient. A paramount example of this
is provided by the Hilbert matrices, known for their bad condition number.
The main consequence of this fact is that the \emph{numerical} solutions to
linear systems of the form $Hx=b$ (where $H$ is a matrix with high condition
number) do not depend continuously on the coefficients of $H$: If $x_0$ is
a solution to $Hx=b_0$ and $x_1$ is a solution to $Hx=b_1$, where $b_0$ and $b_1$
are close in some vector norm, $x_0$ and $x_1$ are \emph{not} necessarily close in
that norm.
In what follows, we focus our attention on computations with Hilbert matrices.
For example:
\begin{verbatim}
(%i142)	F:addcol(hilbert_matrix(5),[137/60,87/60,459/420,743/840,1875/2520]);
(F)	matrix(
		[1,	1/2,	1/3,	1/4,	1/5,	137/60],
		[1/2,	1/3,	1/4,	1/5,	1/6,	29/20],
		[1/3,	1/4,	1/5,	1/6,	1/7,	153/140],
		[1/4,	1/5,	1/6,	1/7,	1/8,	743/840],
		[1/5,	1/6,	1/7,	1/8,	1/9,	125/168]
	)
(%i143)	padic_gauss(F,8209)$
(%i144)	time(%);
(%o144)	[0.161]
(%i145)	padic_backsub(%th(2),8209);
(%o145)	[[[0],0,0,0,0,0,0,0,0],[[0],21,0,0,0,0,0,0,0],
         [[0],8120,8208,8208,8208,8208,8208,8208,8208],
         [[0],141,0,0,0,0,0,0,0],[[0],8140,8208,8208,8208,8208,8208,8208,8208]]
(%i146)	map(lambda([x],hensel_to_farey(x,8209)),%);
(%o146)	[0,21,-89,141,-69]
\end{verbatim}
Observe that \texttt{padic\_gauss} automatically chooses the number $t$ of
digits in the Hensel codes of the solution. There will be a lot of situatons
where our simple heuristics for choosing $t$ will lead to bad values. 
For those cases, one can manually choose $t$ as another (optional) argument. 
Below we choose $8$ digits to solve the system defined by the matrix $E$
defined above, getting the same result as before:
\begin{verbatim}
(%i147)	padic_gauss(E,8209,8)$
(%i148)	padic_backsub(%,8209)$
(%i149)	map(lambda([x],hensel_to_farey(x,8209)),%);
(%o149)	[-1,8,-21,8,20,-19,-3,19,-9,-1]
\end{verbatim}
Let us see how p-adic arithmetic deal with the high condition number of
	Hilbert matrices and the associated instabilities:
\begin{verbatim}
(%i150)	M1:addcol(hilbert_matrix(5),makelist(j,j,1,5));
(M1)	matrix(
		[1,	1/2,	1/3,	1/4,	1/5,	1],
		[1/2,	1/3,	1/4,	1/5,	1/6,	2],
		[1/3,	1/4,	1/5,	1/6,	1/7,	3],
		[1/4,	1/5,	1/6,	1/7,	1/8,	4],
		[1/5,	1/6,	1/7,	1/8,	1/9,	5]
	)
(%i151)	M2:addcol(hilbert_matrix(5),makelist(j+29^(3+random(6)),j,1,5));
(M2)	matrix(
		[1,	1/2,	1/3,	1/4,	1/5,	20511150],
		[1/2,	1/3,	1/4,	1/5,	1/6,	24391],
		[1/3,	1/4,	1/5,	1/6,	1/7,	20511152],
		[1/4,	1/5,	1/6,	1/7,	1/8,	500246412965],
		[1/5,	1/6,	1/7,	1/8,	1/9,	17249876314]
	)
\end{verbatim}
Notice that the non-homogeneous coefficients in $M_2$ are really close to the initial ones in 
$M_1$:
\begin{verbatim}
(%i152)	makelist(padic_distance(M1[j][6],M2[j][6],29),j,1,length(M1));
(%o152)	[1/20511149,1/24389,1/20511149,1/500246412961,1/17249876309]
\end{verbatim}
In this case, the output has a size small enough to be displayed:
\begin{verbatim}
(%i153)	padic_gauss(M1,29,6);
\end{verbatim}
\begin{sideways}
\begin{minipage}{\textheight}
\[
\begin{pmatrix}
\mbox{}
[[0],1,0,0,0,0,0] & [[0],15,14,14,14,14,14] & [[0],10,19,9,19,9,19] & [[0],22,21,21,21,21,21] & [[0],6,23,5,23,5,23] & [[0],1,0,0,0,0,1]\\
[[0],0,0,0,0,0,0] & [[0],17,26,16,26,16,26] & [[0],17,26,16,26,16,26] & [[0],24,26,23,26,23,26] & [[0],2,27,1,27,1,27] & [[0],16,14,14,15,14,28]\\
[[0],0,0,0,0,0,0] & [[0],0,0,0,0,0,0] & [[0],5,19,14,9,24,28] & [[0],22,28,21,28,21,28] & [[0],21,28,20,28,20,28] & [[0],6,24,4,23,4,1]\\
[[0],0,0,0,0,0,0] & [[0],0,0,0,0,0,0] & [[0],0,0,0,0,0,0] & [[0],20,8,11,17,2,26] & [[0],11,17,22,5,5,23] & [[0],5,10,4,28,15,24]\\
[[0],0,0,0,0,0,0] & [[0],0,0,0,0,0,0] & [[0],0,0,0,0,0,0] & [[0],0,0,0,0,0,0] & [[0],16,0,3,1,19,1] & [[0],9,20,8,28,16,9]
\end{pmatrix}
\]
\end{minipage}
\end{sideways}
\begin{verbatim}
(%i157)	padic_backsub(%,29);
(%o157)	[[[0],9,4,0,19,18,1],[[0],20,16,25,14,20,3],
         [[0],19,6,17,8,15,19],[[0],10,20,28,24,27,18],
         [[0],6,21,15,15,0,11]]
(%i158)	map(lambda([x],hensel_to_farey(x,29)),%);
(%o158)	[1/73774969,2/122364169,-4725/9013,1/83362185,2/80199829]
\end{verbatim}
The solutions of the original system and the perturbed one, are quite close
(in the $p-$adic distance, of course):
\begin{verbatim}
(%i159)	makelist(padic_distance(%[j],%th(4)[j],29),j,1,length(%));
(%o159)	[1/24389,1/24389,1/24389,1/24389,1/24389]
\end{verbatim}
For comparison, let us see how things work in the purely rational case:
\begin{verbatim}
(%i160)	A:hilbert_matrix(5)$
(%i161)	X:makelist(x[k],k,1,5)$
(%i162)	B1:makelist(k,k,1,5)$
(%i163)	B2:makelist(rat(k+random(0.01)),k,1,5)$
(%i164)	for j:1 thru 5 do eq1[j]:sum(A[j,k]*X[k],k,1,5)=B1[j]$
(%i165)	for j:1 thru 5 do eq2[j]:sum(A[j,k]*X[k],k,1,5)=B2[j]$
(%i166)	solve(makelist(eq1[j],j,1,5),X);
(%o166)	[[x[1]=125,x[2]=-2880,x[3]=14490,x[4]=-24640,x[5]=13230]]
(%i167)	solve(makelist(eq2[j],j,1,5),X),numer;
(%o167)	[[x[1]=120.1003883272046,x[2]=-2780.48089521768,
          x[3]=14040.16509590324,x[4]=-23939.74158066582,
          x[5]=12880.11148768633]]
\end{verbatim}
The solutions \texttt{(\%o166)} and \texttt{(\%o167)} are far away 
from each other in the usual absolute 	value distance. This example
is a striking evidence for the better stability 	properties of $p-$adic
arithmetic over the traditional rational one (see \cite{7,4,3,5,6,11}).

\begin{thebibliography}{99}
\bibitem{1} G. Bachman: \emph{Introduction to $p-$adic Numbers and Valuation Theory}, 
Academic Press, New York, NY, 1964.
\bibitem{9} K. Conrad: \emph{Hensel's Lemma}. Available at
\url{https://kconrad.math.uconn.edu/blurbs/gradnumthy/hensel.pdf}.
\bibitem{10} J. E. Cremona: \emph{Introduction to Number Theory Lecture Notes}, 2018.
Available at \url{http://homepages.warwick.ac.uk/staff/J.E.Cremona/courses/MA257/ma257.pdf}.
\bibitem{12} F. Q. Gouv\^ea: \emph{$p-$adic Numbers} (2nd. Ed.), Springer Verlag, New York, 2003
\bibitem{7} R. T. Gregory and E. V. Krishnamurthy: \emph{Methods and Applications
of Error-Free Computation}. Springer Verlag, 1984.
\bibitem{2} C. K. Koc: \emph{A Tutorial on $p-$adic Arithmetic}. Oregon State University. 
Technical Report, April 2002. Available at
\url{http://citeseerx.ist.psu.edu/viewdoc/download?doi=10.1.1.81.2931&rep=rep1&type=pdf}.
\bibitem{8} V. E. Krishnamurthy : \emph{On the Conversion of Hensel Codes to Farey
Rationals}. IEEE Transactions on computers, \textbf{C32} 4 (1983).
\bibitem{4} E. V. Krishnamurty, T. Mahadeva Rao and K. Subramanian: \emph{$p-$Adic
arithmetic procedures for exact matrix computations}. Proc, Indian Acad.
Sci., Vol. \textbf{82A} 5 (1975) 165--175.
\bibitem{3} C. Limongelli and R. Pirastu: \emph{$p-$adic Arithmetic and Parallel Symbolic
Computation: An Implementation for Solving Linear Systems}.
Computers and Artificial Intelligence \textbf{14} 1 (1996) 35--62.
\bibitem{5} C. Limongelli: \emph{On an efficient algorithm for big rational number
computations by parallel $p-$adics}. J. Symbolic Computation \textbf{15} (1993) 181--197.
\bibitem{6} C. Lu: \emph{A Computational Library Using p-adic Arithmetic for Exact
Computation With Rational Numbers in Quantum Computing}. Final 
Report (Grant \#FA9550-05-1-0363) Towson University, 2005. Available at
\url{https://archive.org/details/DTIC_ADA456488/page/n5}.
\bibitem{11} A. Vimawala: \emph{$p-$Adic Arithmetic Methods for Exact Computations
of Rational Numbers}. School of Electrical Engineering and Computer Science,
Oregon State University. Available at:
\url{http://cs.ucsb.edu/koc/cs290g/project/2003/vimawala.pdf}.
\end{thebibliography}
\end{document}